% $Id: resume-us.tex,v 2.4 2003/04/24 03:55:41 forman Exp forman $
%-----------------------------------------------------------------------------
\RequirePackage{ifthen}
\newcommand{\ite}{\ifthenelse}
\newcommand{\bl}{\boolean}
%-----------------------------------------------------------------------------
\newboolean{ref}
\setboolean{ref}{false}
%-----------------------------------------------------------------------------
% http://www.uwm.edu/Dept/CDC/resume-develop.htm
% http://www.mnworkforcecenter.org/cjs/cjs_site/index.htm
%
% Tips:
%   - Resumes are not literary, they are promotional. 
%   - The rules of grammar are modified from formal writing. 
%   - Complete sentences are not necessary. 
%   - Avoid the use of "I", as the subject of the resume is assumed to 
%     be the person named in the heading of the resume. 
%   - Avoid long narratives and use lots of bullets and key phrases. 
%   - Someone looking at the resume should be able to figure out the 
%     content without reading the details. 
%   - The resume should draw the reader's attention and create a 
%     desire to know more. 
%   - The goal is to win an interview. 
%   - It is at the interview that the job is won. 
%   - The resume is like a preview of coming attractions. 
%   - You want to save the best for the presentation. 
%   - Therefore, it is best if the resume creates questions in the 
%     mind of the reader.
%
%-----------------------------------------------------------------------------
\documentstyle[margin,line]{res}

%-----------------------------------------------------------------------------
\pdfoutput=1
\relax

%\pdfpagewidth=210mm		%-- A4
%\pdfpageheight=297mm		%-- A4
%\textwidth=145mm		%-- A4
%\textheight=254mm		%-- A4

\pdfpagewidth=8.5in		%-- Letter
\pdfpageheight=11in		%-- Letter
\textwidth=5.75in		%-- Letter
\textheight=9.5in		%-- Letter
 
\oddsidemargin -15mm
\evensidemargin -15mm

\pdfcompresslevel=0             %-- 0 = none, 9 = best
\pdfinfo{                       %-- Info dictionary of PDF output
  /Author (David W. Hodo)
  /Title (Curriculum Vitae)
  /Subject (Curriculum Vitae)
  /Creator (pdfLaTeX)
  /Producer ($Id: resume-us.tex,v 2.4 2003/04/24 03:55:41 forman Exp forman $)
}
\pdfcatalog{          %-- Catalog dictionary of PDF output.
  %/PageMode
  %/UseNone
  %/UseOutlines
  %/UseThumbs
  %/FullScreen
} openaction goto page 1 {/Fit}
%-----------------------------------------------------------------------------
\font\attand=cmr7
\def\Cplusplus{{\rm C\raise.5ex\hbox{\small ++}}}
%-----------------------------------------------------------------------------

\begin{document}

\name{{\Large \bf David W. Hodo}}

\begin{resume}
%\address{
%}
\hspace{-0cm}
\begin{minipage}{1.25\linewidth}
  \begin{minipage}{0.49\linewidth}
    Department of Electrical and Computer Engineering \\
    200 Broun Hall \\
    Auburn University, AL 36849\\
  \end{minipage}%
  \hspace{\fill}%
  \begin{minipage}{0.49\linewidth}
  	\vspace{-4mm}
    Office:\ 334-844-3267 \\
    Mobile:\ 334-332-2386 \\
	{\tt hododav@auburn.edu}
  \end{minipage}
\end{minipage}

%---------------------------------------------------------------------------
\section{\sc Education}
Ph.D., Electrical Engineering, May 2012 (Expected); GPA: 4.00 \\
Auburn University, Auburn, AL

M.S., Electrical Engineering, August 2007; GPA: 4.00 \\
Auburn University, Auburn, AL

B.E.E., Electrical Engineering, December 2005; GPA: 3.91; \emph{Summa Cum Laude} \\
Auburn University, Auburn, AL \\
Minor: Business Engineering Technology\\
Minor: Computer Science
%---------------------------------------------------------------------------

\section{\sc Research Interests}
Unmanned vehicle navigation and control, trailer/implement control, Simultaneous Localization and Mapping (SLAM), state space controls, Kalman Filter design

\vskip 2em

\section{\sc Work Experience}

\begin{format}
\title{l}\employer{r}\\
\location{l}\dates{r}\\
\body\\
\end{format}

%---------------------------------------------------------------------------

\title{\em Staff Engineer}
\employer{\bf Integrated Solutions for Systems (IS4S)}
%\dates{January 2006 -- Present}
\location{Auburn, AL}
\dates{ }
\section{ } %-- twixt \location{} & \begin{position}

\begin{position}
\vspace{-0.9cm}

\section{\sc \small \hspace{8mm}Dec 2011 --\\\hspace{8mm}present}
Project Sponsor: General Dynamics Robotic Systems. Developed an indoor navigation system for a small unmanned ground vehicle (SUGV).  Developed a Simultaneous Localization and Mapping (SLAM) algorithm based on an Extended Kalman Filter (EKF) that blends measurements from wheel odometry, a MEMS inertial measurement unit (IMU), and a laser range finder to provide a position with bounded error in structured indoor and urban environments.

\end{position}

\title{\em Graduate Research Assistant}
\employer{\bf Auburn University}
%\dates{January 2006 -- Present}
\location{GPS and Vehicle Dynamics Laboratory}
\dates{ }
\section{ } %-- twixt \location{} & \begin{position}

\begin{position}
\vspace{-0.9cm}

\section{\sc \small \hspace{8mm}Sept 2009 --\\\hspace{8mm}Dec 2011}
Project Sponsor: General Dynamics Robotic Systems. Developed an indoor navigation system for a small unmanned ground vehicle (SUGV).  Developed a Simultaneous Localization and Mapping (SLAM) algorithm based on an Extended Kalman Filter (EKF) that blends measurements from wheel odometry, a MEMS inertial measurement unit (IMU), and a laser range finder to provide a position with bounded error in structured indoor and urban environments.

\section{\sc \small \hspace{8mm}Apr 2006 --\\\hspace{8mm}Present}
Project Sponsor: U.S. Army Corp of Engineers.  Lead engineer developing unmanned, self-guided wheeled mobile robot towing a trailer that semi-autonomously maps locations of unexploded ordnance on formerly used defense sites. Responsible for all aspects of the system development including path planning, control, navigation, software architecture, and hardware selection.  Developed control system so that geophysical sensors towed by the robot accurately follow a specified path.  Position information is provided by a GPS/INS system coupled with an optical encoder that provides the angle between the robot and sensor trailer.  The tow vehicle used is a Segway RMP400 robot.  The system is capable of being remotely operated and monitored through a custom built operator control unit (OCU) at ranges of up to 1.5 miles.  The vehicle has been demonstrated by Army Corp contractors on several real-world sites.

\end{position}

%
%\title{\em Undergraduate Research Assistant, \\GPS and Vehicle Dynamics Laboratory}
%\employer{\bf Auburn University\\Auburn, AL}
%\dates{\bf September 2005 -- December 2005}
%\location{}
%\begin{position}
%did absolutely noting
%
%\end{position}

\title{\em Intern}
\employer{\bf Northrop Grumman}
%\dates{June 2005 -- August 2005}
\location{Space Technology}
\dates{Warner Robins, GA}

\begin{position}
\vspace{-0.9cm}
\section{\sc \small \hspace{8mm}June 2005 -- \\\hspace{8mm}August 2005}
Assisted in the development of a database driven hardware-in-the-loop simulation system to be used to validate fidelity of various sensors and flight hardware for military aircraft. Developed a graphical user interface so that the simulation system can be easily and quickly configured for different aircraft and sensor configurations.

\end{position}

\title{\em Undergraduate Research Assistant}
\employer{\bf Auburn University}
%\dates{Feb 2004 -- May 2005}
\location{Materials Processing Center}
\dates{ }
\begin{position}
\vspace{-0.9cm}

\section{\sc \small \hspace{8mm}Feb 2004 -- \\\hspace{8mm}May 2005}
Designed a computer based user interface for an experiment to be flown on the International Space Station. The experiments determine material properties of various metals used in the casting industry.  Provided electronics support for various other projects at the center.  Designed and developed a microcontroller based stepper motor speed controller for an experiment to study the effects of particle cohesion in low gravity environments.  
%Also responsible for repairing and operating various electronic equipment including a three dimensional (3-D) scanner and several rapid prototypers, which are used to create scale models of objects designed using computer based tools.

%\newpage
%\opening

%\section{\sc \small \hspace{8mm}Feb 2004 -- \\\hspace{8mm}Aug 2004}
%Designed a computer based user interface for an experiment to be flown on the International Space Station; the experiments determine material properties of various metals used in the casting industry.  Developed software to provide an intuitive interface between NASA's telemetry system and the scientist or engineer running the experiment.

\end{position}


%\title{\em Intern}
%\employer{\bf Russell Corporation}
%%\dates{June 2003 -- August 2003} 
%\location{Process Improvement}
%\dates{Alexander City, AL} 
%\begin{position}
%\vspace{-0.9cm}
%
%\section{\sc \small \hspace{8mm}June 2003 --  \\\hspace{8mm}Aug 2003}
%Assisted the manager of Process Improvement in improving the efficiency of knitting operations.  Conducted a study on knitting machine faults and their relationship to yarn quality.  Presented data and analysis to plant and company managers at regular quality meetings.
%
%\end{position}

%===========================================================================
\newpage
\opening

%\vskip 2em

\section{\sc Teaching Experience}

\begin{format}
\title{l}\employer{r}\\
\location{l}\dates{r}\\
\body\\
\end{format}

%---------------------------------------------------------------------------

\title{\em Graduate Teaching Assistant}
\employer{\bf Auburn University}
%\dates{Spring 2004, Fall 2004}
\location{Electrical Engineering Department}
\dates{ }
\begin{position}
\vspace{-0.9cm}

\section{\sc \small Summer 2008,  Fall 2008, Summer 2009}
Primary instructor for Linear Signals and Systems (ELEC 2120).  Topics covered included system modeling using differential equations, Fourier Series, Fourier Transforms (continuous and discrete), and Laplace Transforms.
\end{position}

%\title{\em Undergraduate Teaching Assistant}
%\employer{\bf Auburn University}
%%\dates{Spring 2004, Fall 2004}
%\location{Electrical Engineering Department}
%\dates{ }
%\begin{position}
%\vspace{-0.9cm}

%\section{\sc \small \hspace{8mm}Spring 2004,  \\\hspace{8mm}Fall 2004}
%Supervised a lab section of the introduction to electrical engineering course (ENGR 1110).
%\end{position}


\section{\sc Key Skills}
\begin{itemize}
	\item Control system and estimator design, embedded system design
	\item Leadership and cross-functional teamwork skills learned through AU Business Engineering Technology minor
  %\item Software: Windows, Linux, M\small{ATLAB}, PSPICE, \LaTeX
  \item Computer Languages: C, C++, Visual Basic 6 and .NET, C\#, M\small{ATLAB}, \LaTeX
\end{itemize}

\section{\sc Professional Societies}
%\begin{itemize}
%	\item 2005 -- Present: Institute of Electrical and Electronic Engineers
%\end{itemize}
2005 -- Present: Institute of Electrical and Electronic Engineers

\vskip 2em

\section{\sc Honors}
\begin{itemize}
  \item Eta Kappa Nu (HKN) -- Electrical Engineering Honor Society.
  \item Tau Beta Pi (TB$\Pi$) -- National Engineering Honor Society.
 % \item Golden Key National Honor Society.
  \item AU Dean's List: 2002, 2003, 2004.
\end{itemize}

%------------------------------------------------------------------------------
\section{\sc Publications}
D.~W.~Hodo, D.~M.~Bevly, J.~Y.~Hung, S.~Millhouse, B.~Selfridge, ``Optimal Path Planning with Obstacle Avoidance for Autonomous Surveying.'' \emph{Proceedings of the 36th Annual Conference of the IEEE Industrial Electronics Society}, Pheonix, AZ, November 2010.

D.~W.~Hodo. ``Ch. 6 Vehicle Control.'' \emph{GNSS for Vehicle Control.} Ed. D. Bevly and S. Cobb. Artech House: Boston, 2010.	

W.~Travis, S.~Martin, D.~W.~Hodo, D.~Bevly, ``Non-Line of Sight  Automated Vehicle Following Using a Dynamic Base RTK System.'' Accepted to \emph{Navigation: Journal of the Institute of Navigation}.

N.~Harrison, B.~Selfridge, C.~Murray, and D.~Hodo, ``Self-guiding robotic geophysical surveying for shallow objects in comparison to traditional survey methods,'' Presented at the Symposium on the Application of Geophysics to Environmental and Engineering Problems (SAGEEP), Keystone, Colorado, April 2010.

N.~Harrison, B.~Selfridge, M.~Root, C.~Murray, D.~Hodo, D.~S.~Millhouse,  ``Self-Guiding Robotic System Surveying and Comparison to Traditional Survey Methods.'' \emph{Proceedings of the UXO / Countermine / Range Forum� 2009}, Orlando, FL, August 2009.

W.~Travis, D.~W.~Hodo, D.~M.~Bevly, and J.~Y.~Hung, ``UGV trailer position estimation using a dynamic base RTK system,'' \emph{Proceedings of the 2008 AIAA Guidance, Navigation and Control Conference}, Honolulu, HI, Aug 2008.

D.~W.~Hodo, ``Development of an autonomous mobile robot-trailer system for UXO detection,'' Master's thesis, Auburn University, August 2007.

D.~W.~Hodo, J.~Y.~Hung, D.~M.~Bevly, S.~Millhouse, ``Linear Analysis of Trailer Lateral Error with Sensor Noise for a Mobile Robot-Trailer System.'' \emph{Proceedings of the 2007 IEEE International Symposium on Industrial Electronics}, Vigo, Spain, June 2007.

D.~W.~Hodo, J.~Y.~Hung, D.~M.~Bevly, S.~Millhouse, ``Effects of Sensor Placement and Errors on Path Following Control of a Mobile Robot-Trailer System.'' \emph{Proceedings of the 26th Annual American Controls Conference}, New York City, July 2007.


\end{resume}

\end{document}
